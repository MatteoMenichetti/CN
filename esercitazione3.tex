\section{Esercitazione capitolo 3}
\subsection{A.A. 2022/23}
\begin{enumerate}
    \item  Scrivere una function Matlab che risolva efficientemente un sistema triangolare inferiore.
    \item Definire la fattorizzazione $LU$ di una matrice nonsingolare. Dimostrare che, se fattorizzabile, la fattorizzazione $LU$ di una matrice è unica.
    \item  Sotto quali condizioni esiste la fattorizzazione $LU$ di una matrice?
    \item Definire cosa si intende per matrice a diagonale dominante. Dimostrare che una matrice diagonale dominante è fattorizzabile $LU$.
    \item  Definire cosa si intende per matrice simmetrica e definita positiva (sdp). Dimostrare che una matrice sdp è fattorizzabile $LU$ .
    \item Dimostrare che una matrice simmetrica e definita positiva è fattorizzabile nella forma $LDL^T$, con $L$ triangolare inferiore a diagonale unitaria, e $D$ matrice diagonale.
    \item  Definire cosa si intende per norma indotta su matrice. Dare qualche esempio di tali norme.
    \item  Definire cosa è il numero di condizionamento di una matrice. Spiegarne il significato.
    \item Definire la soluzione nel senso dei minimi quadrati di un sistema lineare sovra-determinato a rango pieno. Dimostrarne l’esistenza ed unicità.
    \item Calcolare la matrice elementare di Householder $H$ relativa al vettore: 
    \begin{equation*}
       \boldsymbol z = \begin{pmatrix}
        -1\\
        2\\
        -2
    \end{pmatrix}.
    \end{equation*}
    Quanto vale il prodotto $H\boldsymbol z$?
    \item Definire il metodo di Newton per sistemi nonlineari, e dettagliarne l’implementazione.
\end{enumerate}

\paragraph{1.}\footnote{Slide 3 PDF 14.} La function è implementata nell'Algoritmo \ref{alg:trilow}.

\begin{algorithm}
\caption{Implementazione efficiente risolutore sistema triangolare inferiore.}\label{alg:trilow}
    \begin{lstlisting}[style=Matlab-editor]
    function y = trilow(L, b)
    %   
    %   y = trilow(L, b)
    %
    %   Risolve un sistema triangolare inferiore
    %
    % Input:
    %   L   -  matrice dei coefficienti;
    %   b  -   termine noto;
    %   
    % Output:
    %   y   -   vettore soluzione.
    %
    % Le prossime tre righe sono controlli iniziali.
    [m,n] = size(L);
    if m ~= n, error('Matrice non quadrata'), end
    if n~= length(b), error('Termine noto non consistente'), end

    y=b(:); %trasformazione di b in vettore colonna
    
    for i = 1 : n
        if L(i,i) == 0, error('Matrice singolare'), end
        y(i) = y(i)/L(i,i);
        y(i+1:n) = y(i+1:n) - L(i+1:n, i)*y(i):
    end
    return
    \end{lstlisting}
\end{algorithm}

\paragraph{2.}\footnote{Slide 3 PDF 14.} Sia $A\in\mathbb R^{n\times n},\, \det(A).$

\noindent (Risposta alla prima parte di domanda:) $A$ è fattorizzabile $LU$, se $A=LU$, con:
\begin{itemize}
    \item $L$ triangolare inferiore a diagonale unitaria:
    \item $U$ triangolare superiore.
\end{itemize}

\noindent (Risposta alla seconda parte) Unicità della fattorizzazione: È necessario dimostrare che se $A=L_1U_1$ è un'ulteriore fattorizzazione $LU$ di $A$, allora $L=L_1,\, U=U_1.$ Infatti:
\begin{equation*}
    0\neq\det(A)=\det(LU)=\underbrace{\det(L)}_{\footnotemark}\det(U)=\det(U).
\end{equation*}
\footnotetext{Uguale ad 1 perché a diagonale unitaria 1.}

\noindent Pertanto, da $LU=L_1U_1$, segue che, moltiplicando a destra per $U^{-1},\, L=L_1U_1U^{-1}.$ Da questa, moltiplicando a sinistra per $L_1^{-1},$ segue che:
\begin{equation}\label{eq:LL=UU}
    L_1^{-1}L=U_1U^{-1}.
\end{equation}

\noindent È possibile osservare ciò che segue:
\begin{itemize}
    \item $L_1^{-1}$ è una matrice triangolare inferiore a diagonale unitaria;
    \item  $U^{-1}$ è una matrice triangolare superiore.
\end{itemize}

\noindent Quindi $L_1^{-1}L$ è una matrice triangolare inferiore a diagonale unitaria e $U_1U^{-1}$ è una matrice triangolare superiore. Pertanto, le due matrici in (\ref{eq:LL=UU}) sono diagonali. Poichè la diagonale di $L_1^{-1}L$ è unitaria, questa è la matrice indentità. Dalle uguaglianze allora 
\begin{itemize}
    \item $L_1^{-1}L=I$,
    \item $U_1U^{-1}=I$,
\end{itemize}
segue che $L=L_1\wedge U_1=U$, ovvero la fattorizzazione $LU$ è unica. \qed

\paragraph{IMPORTANTE:} I passaggi come quelli sopra è necessario che siano il piu' dettagliati possibile affinché Luigi sappia che abbiamo studiato.

\paragraph{3.} Sia $A\in\mathbb R^{n\times n}, \det(A)\neq 0.$ Denotando con $A_k$ la sottomatrice principale di ordine $k$ di $A$, allora:
\begin{equation*}
    A=LU\iff\forall k=1,\hdots, n\, :\,\det(A_k)\neq 0.
\end{equation*}
Questa ultima riga non è sufficiente, è necessario specificare $A\in\mathbb R^{n\times n}.$

\paragraph{4.}\footnote{Slide 6 PDF 14.} Sia $\boldsymbol{A\in\mathbb R^{n\times n}}$. Allora $A=(a_{ij})$ è:
\begin{itemize}
    \item \textbf{diagonale dominante per righe}, se
    \begin{equation}\label{eq:ddRighe}
        |a_{ii}|>\sum_{j\neq i, j=1}^n|a_{ij}|,\quad\forall i=1,\hdots,n;
    \end{equation}
    \item \textbf{diagonale dominante per colonne}, se $|a_{jj}|>\sum_{i\neq j, i=1}^n |a_{ij}|,\;\forall j=1,\hdots,n.$
\end{itemize}
Valgono le seguenti proprietà:
\begin{itemize}
    \item[\textbf{P1)}] $A$ è d.d. per righe $\iff A^T$ è d.d. per colonne;
    \item[\textbf{P2)}] Se $A$ è d.d. per righe (o colonne), allora, detta $A_k$ la sottomatrice principale di ordine $k$ di $A$, essa sarà d.d. per righe (o colonne). (Dimostrazione:) Infatti, dato un generico $k\in\{1,\hdots,n\},$ risulterà che dalla (\ref{eq:ddRighe}) segue che $\forall i=1\hdots,k:$
    \begin{equation*}
         |a_{ii}|>\sum_{j\neq i, j=1}^n|a_{ij}|\geq \sum_{j\neq i, j=1}^k|a_{ij}|,
    \end{equation*}
    ovvero, $A_k$ è diagonale d.d. per righe. Analogamente per $A$ d.d. per colonne.\qed
    \item[\textbf{P3)}] Se $A$ è d.d. per righe (o per colonne), allora $A$ è non singolare. Infatti, se $A$ fosse singolare, esisterebbe $\uline{x}\in\mathbb R^n,\, x\neq 0\,:\, A\uline{x}=\uline 0$. Questa equazione vale per ogni multiplo di $\uline x$ quindi è possibile normalizzare $\uline x$ in modo che $x_k=\underset{i=1,\hdots,n}{\max}|x_i|=1$. Pertanto, la $k$-esima equazione di $A\uline x=\uline 0$ diviene: $\sum_{j=1}^na_{kj}x_j=0.$ Da questo segue che $a_{kk}\,x_k=-\sum_{j=1,j\neq k}^n a_{kj}x_j.$ Pertanto:
    \begin{equation*}
        |a_{kk}|=|a_{kk}x_k|=\left|\sum_{j=1,j\neq k}^n a_{kj}x_j\right|\leq\sum_{j=1,j\neq k}^n |a_{kj}|\cdot\underbrace{|x_j|}_{\leq 1}\leq\sum_{j=1,j\neq k}^n |a_{kj}|,
    \end{equation*}
    il che contradice l'ipotesi d.d. per righe sulla riga $k$. Se A è d.d. per colonne, il tutto è riapplicato a $A^T$, che sarà d.d. per righe (per \textbf{P1)}).
\end{itemize}

\noindent Dalle proprietà \textbf{P2)} e \textbf{P3)}, segue che:
\begin{itemize}
    \item $A$ è d.d. per righe (o colonne) $\iff\forall k=1,\hdots,n;$
    \item $A_k$ è d.d. per righe (o colonne) $\iff\forall k=1,\hdots,n;$
    \item $\det(A_k)\neq 0\iff A$ è fattorizzabile $LU.$
\end{itemize}

\paragraph{5.}\footnote{Slide 9 PDF 14.} Sia $A\in\mathbb R^{n\times n}.$ A è simmetrica e definita positiva (sdp) se:
\begin{itemize}
    \item $A=A^T$ ($A$ è simmetrica);
    \item $\forall \uline x\in\mathbb R^n\backslash\{\uline 0\}\,:\, \uline x^T A \uline x > 0$ ($A$ è definita positiva).
\end{itemize}
Per dimostrare che, se $A$ è sdp, allora $A=LU$, sarà dimostrato che:
\begin{enumerate}
    \item $\forall k=1,\hdots, n$, $A_k$, detta sottomatrice principale di $A$ di ordine $k$, è sdp;
    \item $A$ sdp $\Rightarrow A$ è nonsingolare.
\end{enumerate}

\noindent Riguardo la 2), se $A$ non fosse singolare, allora $\exists\,\uline x\in\mathbb R^n\backslash\{\uline 0\}\, : \, A\uline x=\uline 0\Rightarrow\uline x^T A \uline x = 0,$ contraddicendo la definita positiva di A.
Pertanto, $A$ è nonsingolare.

\noindent Riguardo la 1): 
\begin{equation}\label{eq:ACompSimm}
\forall k=1,\hdots,n:\; 
A=\left[\begin{array}{c|c}
       A_k & B\\
       \hline
       C& D
    \end{array}
    \right],\text{ con } D\in\mathbb R^{(n-k)\times (n-k)}.    
\end{equation}
Dalla simmetria di $A$, segue che 
$A^T=\left[
\begin{array}{c|c}
    A_k & C^T\\
    \hline
    B^T & D^T
\end{array}
\right]=A.$ Uguagliando i blocchi omologhi allora:
\begin{equation*}
    A_k=A_k^T,\; B=C^T,\; D=D^T\;\rightarrow A_k \text{ è simmetrica.}
\end{equation*}

\noindent Rimane da dimostrare che, $\forall \uline y\in\mathbb R^k\backslash\{\uline 0\}\,:\, \uline y^TA_k\uline y > 0.$ A questo fine, assegnato un generico $\uline y\in\mathbb R^k\backslash\{\uline 0\}$, è considerato il vettore $\uline x=
\begin{pmatrix}
    \uline y\\
    \uline 0
\end{pmatrix}\in\mathbb R^n$. Chiaramente $\uline x\neq 0$. Dato che $A$ è sdp allora da (\ref{eq:ACompSimm}) segue che:
\begin{equation*}
    0<\uline x^T A \uline x = (\uline y^T\; \uline 0^T)\,\left[\begin{array}{c|c}
       A_k & B\\
       \hline
       C& D
    \end{array}
    \right]
    \begin{pmatrix}
        \uline y\\
        \uline 0
    \end{pmatrix} = 
    (\uline y^T\; \uline 0^T)
    \begin{bmatrix}
        A_k \uline y\\
        C \uline y
    \end{bmatrix} = \uline y^T A_k \uline y.
\end{equation*}\qed

\paragraph{6.}\footnote{Slide 11 PDF 14.}
È noto che, se $A$ è sdp, allora $A=LU$ ed è noto che la fattorizzazione $LU$ è unica. Pertanto, osservando il fattore $U$ può essere scritto come $D\widehat{U}$, con $D$ diagonale ed $\widehat{U}$ triangolare superiore a diagonale unitaria, segue che $A=LU=LD\widehat{U}$. Essendo $A=A^T\Rightarrow A=LD\widehat{U}=(LD\widehat{U})^T=A^T.$ Segue che $LD\widehat{U}=\widehat{U}^TDL^T.$

\noindent Essendo:
\begin{itemize}
    \item $\widehat{U}^T$ triangolare inferiore a diagonale unitaria,
    \item $DL^T$ triangolare superiore,
\end{itemize}
per l'unicità della fattorizzazione $LU$ segue che $L=\widehat{U}^T$. Da questo è possibile concludere che $A=LDL^T.$\qed

\paragraph{7.}\footnote{Slide 12 PDF 14.} Sia $||\cdot||$ una norma assegnata su vettore. È definito, data $A\in\mathbb R^{m\times n}$, la sua norma indotta dalla norma su vettore considerata: $||A||=\underset{\uline x\in\mathbb R^n, ||x||=1}{\max}||A\uline x||$.

\begin{remark}
    $||A\uline x||$ è la norma di un vettore di $\mathbb R^m.$
\end{remark}

\noindent Alcuni esempi: se $A=(a_{ij})\in\mathbb R^{m\times n},$ allora:
\begin{itemize}
    \item $||A||_1=\underset{j=1,\hdots,n}{\max}\sum_{i=1}^m|a_{ij}|;$
    \item $||A||_\infty=\underset{i=1,\hdots,m}{\max}\sum_{j=1}^n|a_{ij}|.$
\end{itemize}

\paragraph{8.}\footnote{Slide 12 PDF 14.} Sia $A\in\mathbb R^{n\times n},\;\det(A).$ Il suo numero di condizione, assegnato ad una generica norma indotta da una corrispondente norma su matrice, è definito come $\kappa(A)=||A||\cdot||A^{-1}||.$ Questo misura il condizionamento del sistema lineare $A\uline x= \uline b.$ Infatti, considerando perturbazioni $\Delta A$ e $\Delta\uline b$ su $A$ e $\uline b$, rispetivamente, è ottenuto $(A+\Delta A)(\uline x+\Delta\uline x)= \uline b+\Delta\uline b,$ con $\Delta\uline x$ perturbazione sul risultato. Vale che:
\begin{equation*}
    \frac{||\Delta\uline x||}{||\uline x||}\leq\kappa(A)\left(\frac{||\Delta\uline b||}{||\uline b||}+\frac{||\Delta A||}{||A||}\right)\qed
\end{equation*}

\paragraph{9.}\footnote{Slide 13 PDF 14.} Sia $A\in\mathbb R^{m\times n},$ con $m>n=rank(A).$ È definita \textbf{soluzione nel senso dei minimi quadrati} del sistema lineare $\boldsymbol{A\uline x=\uline b},$ il vettore $\boldsymbol{\uline x}$ \textbf{che minimizza la norma euclidea (al quadrato) del corrispondente vettore residue} $\boldsymbol{\uline r=A\uline x-\uline b}.$ \footnote{Necessario specificare la parte in grassetto affinché la risposta sia corretta.}

\noindent Il motivo della scelta della norma euclidea è dovuto al fatto che questa è invariante per moltiplicazione del vettore in argomento per una matrice ortogonale $Q$:
\begin{equation*}
    ||Q\uline v||_2^2=(Q\uline v)^T(Q\uline v)=\uline v^T \underbrace{Q^TQ}_{\footnotemark}=\uline v^T\uline v=||\uline V||_2^2.
\end{equation*}\footnotetext{Uguale ad $I$ perché $Q$ è ortogonale.}
La soluzione ai minimi quadrati di $A\uline x=\uline b$ è ottenuta osservando che, se $A\in\mathbb{m\times n},\; m>n=rank(A)$, allora esistono:
\begin{itemize}
    \item $Q\in\mathbb R^{m\times m},\; Q$ ortogonale ;
    \item $\widehat{R}\in\mathbb R^{n\times n},\;\widehat{R}$ triangolare superiore e nonsingolare,
\end{itemize}
tali che.: $A=QR,$ con $R=\begin{pmatrix}
    \widehat{R}\\
    0
\end{pmatrix}\in\mathbb R^{m\times n}.$ Per minimizzare:
\begin{equation*}
    \begin{matrix}
        ||\uline r||_2^2&=&||A\uline x-\uline b||_2^2&=&||QR\uline x-\uline b||_2^2 &=& ||Q(R\uline x - Q^T\uline b)||_2^2&\overset{\footnotemark}{=}&||R\uline x-\uline g||_2^2\\
        &\overset{\footnotemark}{=}&
        \Biggl|\Biggl|\begin{bmatrix}
            \widehat{R}\\
            0
        \end{bmatrix}\uline x
        - \begin{bmatrix}
            \uline g_1\\
            \uline g_2
        \end{bmatrix}\Biggl|\Biggl|_2^2&=&\Biggl|\Biggl|\begin{bmatrix}
            \widehat{R}\uline x -\uline g_1\\
            -\uline g_1
        \end{bmatrix}\Biggl|\Biggl|_2^2 &=& \left|\left|\widehat{R}\uline x - \uline g_1\right|\right|_2^2 + \left|\left|\uline g_2\right|\right|_2^2 &=& \left|\left|\uline g_2\right|\right|_2^2\\
        &&&&&&&=& \min!
    \end{matrix}
\end{equation*}

\addtocounter{footnote}{-1}
\footnotetext{Ponendo $\uline g=Q^T\uline b$ e sfruttando l'invarianza di $||\cdot||_2$.}

\stepcounter{footnote}
\footnotetext{$\uline g_1\in\mathbb R^n$.}

\noindent Scegliendo $\uline{x}$ come soluzione del sistema lineare $\widehat{R}\,\uline x=\uline g_1$, che esiste, ed è unica, essendo $\widehat{R}$ non singolare.

\paragraph{10.}\footnote{Slide 15 PDF 14.} È noto che $H\boldsymbol{\uline z}=\alpha\uline e_1\in\mathbb R^3, $ con $\alpha=\pm\left|\left|\uline z\right|\right|_2=\pm 3.$ La matrice $H$ è nella forma
\begin{equation*}
    H=I-\frac{2}{\uline v^T\uline v}\uline v\uline v^T,
\end{equation*}
con
\begin{equation*}
    \uline v=\uline z-\alpha \uline e_1=
    \begin{bmatrix}
        -1-\alpha\\
        2\\
        -2
    \end{bmatrix}=
    \begin{bmatrix}
        -1-3\\
        2\\
        -2
    \end{bmatrix}
\end{equation*}
avendo scelto \textbf{$\boldsymbol{\alpha=3}$, in modo che la prima componente di $\boldsymbol{\uline v}$ sia ottenuta sommando quantità concordi.} Segue che $\boldsymbol{H\uline z= 3\uline e_1}$. Il vettore di Householder è $\uline v=\begin{bmatrix}
    -4\\
    2\\
    -2
\end{bmatrix},$ dal quale è ottenuto che
\begin{equation*}
    H=I-\frac{2}{\uline v^T\uline v}\uline v\uline v^T=I-\frac{2}{24}\uline v \uline v^T=I-\frac{1}{12}\uline v\, \uline v^T.
\end{equation*}

\paragraph{11.}\footnote{Slide 17 PDF 14.} Sia $f\,:\,\mathbb R^m\rightarrow\mathbb R^m,$ è definito il metodo di Newton per risolvere $f(\uline x)=\uline 0$ fondamentalmente dall'iterazione seguente:
\begin{equation}\label{eq:passoFondamNewt}
    \uline x_{n+1}=\uline x_n - J_f(\uline x_n)^{-1}f(\uline x_n),\; n=0,1\hdots
\end{equation}
essendo $J_f(\uline x_n)$ la matrice Jacobiana di $f(\uline x)$ calcolata in $\uline x_n$. Nella pratica (\ref{eq:passoFondamNewt}) è implementata con l'obbiettivo di risolvere il sistema lineare $J_f(\uline x_n)\Delta \uline x_n=-f(\uline x_n)$ aggiornando $\uline x_{n+1}=\uline x_n+\Delta\uline x_n,\, n=0,1,\hdots\boldsymbol .$ Pertanto, il costo per iterazione è il seguente:
\begin{enumerate}
    \item 1 valutazione di $J_f$ e la sua fattorizzazione;
    \item 1 valutazione di $f$ e soluzione per i fattori di $J_f$;
    \item 1 axpy per aggiornare l'approssimazione corrente.
\end{enumerate}
Come criterio d'arresto è possibile utlizzare il seguente:
\begin{equation*}
    ||\Delta\uline x_n./(1+|\uline x|)||\leq tol, 
\end{equation*}
per una tolleranza $tol$ specificata.